% !TEX root = ../chrysalis-report.tex
%
\chapter{Introduction}
\label{sec:intro}


In times of digitalization, \emph{Business Process Management} (BPM) systems play a central role in efficiently orchestrating and executing business processes automatically. State-of-the-art systems use a centralized program, the BPM \emph{Engine}, to offer this functionality. While this concept has proven itself for single-organization BPM, applying the same central system to a inter-organizational process leads to trust issues: With only one engine controlling all processes with absolute authority, it is difficult to ensure that the administrator of said system doesn't illegally influence the deployed processes for their own gain. \newline 
 The project \emph{Chrysalis} seeks to eliminate this trust issue caused by a centralized BPM engine by instead deploying and running this engine on a Blockchain network, which is naturally decentralized, offers more transparency to it's peers and only allows for changes, if they're approved by a consensus mechanism. \newline
 \emph{Chrysalis} was developed in tandem with the paper \cite{chrysalis} as a proof-of-concept prototype and handed to us for improvement in this project.


\section{Problem Statement}
\label{sec:intro:problem}

\emph{Chrysalis}, in it's initial state, offered the basic functionality of a BPM engine using a browser-based front-end, deploying and utilizing logic on the \emph{Ethereum} blockchain in a private network. However, as it was made to be a prototype, room for expansions and feature introductions hadn't been a priority so far. \newline
Our Task therefore is to refine \emph{Chrysalis} by overhauling it's architecture, extending it's features and following the steps necessary to make the system run efficiently. By that, we intend to make \emph{Chrysalis} more compatible with modern systems and demonstrate that it can fulfil the typical demands of users. \newline

\section{Results}
\label{sec:intro:results}

To reach the aforementioned goals, we were given a set of tasks. First, we overhauled the architecture of the underlying process management library \emph{Enzian-Yellow} to increase maintainability and extensibility, as well as added a persistence layer to permanently store relevant data like the addresses of deployed process instances. Following up on this, we worked on the Blockchain applications more directly, overhauling the \emph{smart contract} code that is deployed on the \emph{Ethereum} network and developing an entirely new application on the \emph{Hyperledger} network, mirroring the functionality of the Ethereum variant. Additionally, we compared the capabilities and efficiency of \emph{Chrysalis}' Ethereum back-end to a similar and more mature system, \emph{Caterpillar}.

All given tasks were completed as specified, with the exception of the \emph{Hyperledger} implementation, which runs successfully, yet has compatibility issues with \emph{Chrysalis}' front-end, which we didn't manage to fix in time. This issue is documented in detail in section \ref{sec:issues:integration}.

\section{Report Structure}
\label{sec:intro:structure}

\textbf{Chapter \ref{sec:init}} \\[0.2em]
In the first content chapter, we will explain the way Chrysalis generally functions. Starting from an abstract and high-level view where the intended uses of the application are explained, we will the continue to detail the components that make up Chrysalis - what their role in the system is, how they interact with each other and with what external components they interface. At last, we will describe how business processes are modeled inside a blockchain node, so that our application may interact with them in a defined way.

\textbf{Chapter \ref{sec:impr}} \\[0.2em]
This chapter being the biggest of all, we intend to describe our programming work done here. This includes all improvements, additions and removals in the code. To do this, for every major task we will first describe its meaning and implications, then give a modeler's overview of the changes done. Where needed, we will give some technical insights to our work. Concluding every task as well as the entire chapter, we will summarize our results.

\textbf{Chapter \ref{sec:issues}} \\[0.2em]
Given that this project wasn't intended to be perfected once our work was done, and also given that some problems arose that hampered the quality of our results, this chapter will describe said issues. We will both clarify where they stem from and propose some ways of solving them, so those who will be handed this project may fix them with relative ease.

\textbf{Chapter \ref{sec:caterpillar}} \\[0.2em]
In this chapter we evaluate our project with respect to another blockchain-based process model execution system named Caterpillar. We will give an overview of its architecture, point out important features of both its compilation-based and interpretation-based versions, as well as compare it with CHRYSALIS both featurewise and performancewise.

\textbf{Chapter \ref{sec:res}} \\[0.2em]
Due to the project having grown in complexity during our work on it, we decided to build a repository of design documents and other helpful files. In this short section, we intend to present these resources.

\textbf{Chapter \ref{sec:conclusion}} \\[0.2em]
Finally, we will evaluate the success of our project and speak about the opportunities and limits it offers to those interested in deploying a decentralized process management engine.
